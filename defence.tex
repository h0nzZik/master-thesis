\documentclass[11pt]{beamer}
\usetheme{Boadilla}
\usepackage[utf8]{inputenc}
\usepackage[czech]{babel}
\usepackage[T1]{fontenc}
\usepackage{amsmath}
\usepackage{amsfonts}
\usepackage{amssymb}
\usepackage{graphicx}
\author{Jan Tušil}
\title{An Executable Formal Semantics of C++}
%\setbeamercovered{transparent} 
%\setbeamertemplate{navigation symbols}{} 
%\logo{} 
\institute{FI MU} 
\date{1.2.2017} 
%\subject{} 
\begin{document}


% Osnova:
% Kdo jsem - Muj obor (PdS), 
% Kontext
% Cile prace
% Co se podarilo
% Dalsi kroky
% Co chci zminit:
% - kdo me vedl, s kym jsem praci konzultoval, UIUC, RV


% Koho mam v komisi? Pelanek, Strejcek, Hlineny. Vsichni trochu do tech formalnich
% metod vidi. Strejda praci cetl, tedy musim se ji snazit prodat hlavne Pelankovi a Hlinenemu.
% Nesmim se poustet do spekulaci.
% K tomu, ze jsem nasel chyby v Gcc a Clang, a nejakou nejednoznacnost ve standardu:
% to neni tim, jaky jsem borec. To se proste stava, kdyz nekdo neco formalizuje. Kdyz lidi
% pracovali na semantice JavaScriptu, take nasli chyby v interpreterech. A ted behem praci
% na formalizaci Etherea.

% K cemu je ta semantika dobra? Jak se to testuje? Mame interpreter, ktery ji pouziva.

% Dobrý den, já jsem Honza Tušil a rád bych vám představil svoji diplomovou práci, 
% ve které jsem se zabýval vytvářením spustitelné formální sémantiky pro jazyk C++.
\begin{frame}
\titlepage
\end{frame}

% Během následující čtvrthodinky nejprve přiblížím kontext mojí práce,
% 
\begin{frame}
\tableofcontents
\end{frame}

\section{Kontext}
% O mě, o C++, K framework (přepisování termů), k cemu to, RV, UIUC
% Moje osobní motivace - ověřování higlevel vlastností programů v C++ (šablony apod).

\begin{frame}{K Framework}

\end{frame}

% Zadání, cíle, konkrétní věci, nalezené chyby apod.
\section{Obsah práce}

% Tady bych jenom dodal, že nalezení chyb v překladačíh
% nebo standardech programovacích jazyků není až tak moc neobvyklé,
% zvlášť v kontextu formalizace těchto jazyků - JavaScript, Ethereum.
\subsection{Chyby nalezené v jiných projektech}

% Co se povedlo, co jsem se naucil, jake jsou moje dalsi plany.

% Možná mohu říci, že práce s K frameworkem sice byla složitá v tom smyslu,
% že jsem se musel naučit hodně nových věcí, vžít se do nového programovacího paradigmatu
% apod, ale nebyla potřeba žádná hluboká matematická znalost (žádné fixpointy apod).
\section{Vyhodnoceni}

% Po otázkách - kdyby chtěli vědět něco dalšího, rád si s nima popovídám.

\end{document}